\chapter{Solving the Shortest Path Problem}
\label{chap:solvingspp}

Over the years,
various algorithms have been developed 
to address the problem of finding the shortest path.
This chapter states notations and definitions for the shortest path problem and discusses
the theory for solving it.
Algorithms that are applicable for road networks are summarised,
including the discussion of their advantages and drawbacks.

\section{Notations and Definitions}
The Shortest Path Problem (SPP) is the problem of finding the shortest path from a given origin  to some destination.
There are two types of SPP that are going to
be analysed in this chapter:
a single-source and a point to point SPP.  
The Frank-Wolfe algorithm in the TA involves
solving the single-source SPP by finding shortest path going from one origin to every other destinations of the network.
The Path Equilibration method in the TA
Solving the point to point SPP solves from one origin to a specific destination and is used in the Path Equilibration method. 

When solving SPP for a normal road network,
different measurements such as distance and travel exist for the road length.
But in traffic assignment,
the road length is measured in a non-decreasing travel time function,
which encapsulates information such as traffic flow, road capacity and travel speed.
This travel time function is always non-negative so taking advantage of this helps the selection of algorithms that uses this property.
\todo{show equation?}

Here we present the notations mainly borrowed from \citet{Cormen} and \citet{Klunder},
we denote $ G = ( V, E ) $ a weighted, directed graph,
where $ V $ is the set of nodes (origins, destinations, and intersections)
and $ E $ the set of edges (roads).
We say $ E $ is a subset of the set $ \{ (u, v)\, | \, u, v \in V \} $ of all ordered pairs of nodes.
We denote the link cost function $ c : E \rightarrow \mathbb{R} $ which assigns a cost (travel time) to any arc $ (u,v) \in E $ depending on traffic flow on that arc.
We write the costs of arc $(u, v)$ as: $ c((u, v)) = c_{uv} $.

The path $P$ inside a transportation network has to be a directed simple path, 
which is a sequence of nodes and edges $ (u_1, (u_1, u_2), u_2, \ldots , (u_{k-1}, u_k), u_k ) $
such that $ (u_i, u_{i+1}) \in E$ for $i = 1,\ldots,k-1$ and $u_i \neq u_j$ for all $ 1 \leq i < j \leq k$.
Note $u_1$ is the origin and $u_k$ is the destination of the path $P$, $u_1$ and $u_k$ together is called an O-D pair for this path.
For simplicity, we denote $s$ to be the source (origin) and $t$ to be the target (destination) for any path $P$.
%Finally we denote cost of the whole path $C(P) := \sum_{(u,v)\in P} c_{vw}$.

In a transportation network,
the origins and destinations are often called centroids or zones.
They are used for generating trip demands and supplies
and hold information such as household income and employment information.
These information helps to understand trips that are produced and attracted within the zone.
The zones are conceptual nodes in the network and are untravellable,
which means a path between two zone nodes must not contain another zone node.

\begin{figure}[H]
    \tikzstyle{main node} = [circle, draw, text centered, minimum height=2.5em]
    \tikzstyle{line} = [->, draw]
    \centering
    \begin{subfigure}[b]{0.4\textwidth}
        \centering
        \begin{tikzpicture}[>=stealth', line width=1pt, auto, node distance=3cm]
            \node[main node] (zone) {zone};
            \node[main node] (1) [below left of=zone] {1};
            \node[main node] (2) [below right of=zone] {2};
        %\path [line, red] (1) -- (zone);
            \path [line, green!80!black] (zone) -- (2);
            \path [line, green!80!black] (1) -- (2);
            %\node (text) [yshift=-5em,above of=zone] {origin node};
        %\node at ($(zone) !.5! (2)$) {$\mathbin{\tikz [x=1.4ex,y=1.4ex,line width=.3ex, black] \draw (0,0) -- (1,1) (0,1) -- (1,0);}$};
        \end{tikzpicture}
        \caption{zone as origin node}
    \end{subfigure}
    \quad
    \begin{subfigure}[b]{0.4\textwidth}
        \centering
        \begin{tikzpicture}[>=stealth', line width=1pt, auto, node distance=3cm]
            \node[main node] (zone) {zone};
            \node[main node] (1) [below left of=zone] {1};
            \node[main node] (2) [below right of=zone] {2};
            \path [line, green!80!black] (1) -- (zone);
        %\path [line, red] (zone) -- (2);
            \path [line, green!80!black] (1) -- (2);
            %\node (text) [yshift=-5em,above of=zone] {destination node};
        %\node at ($(zone) !.5! (2)$) {$\mathbin{\tikz [x=1.4ex,y=1.4ex,line width=.3ex, black] \draw (0,0) -- (1,1) (0,1) -- (1,0);}$};
        \end{tikzpicture}
        \caption{zone as destination node}
    \end{subfigure}

    \vspace{1cm}
    \begin{subfigure}[t]{\textwidth}
        \centering
        \begin{tikzpicture}[>=stealth', line width=1pt, auto, node distance=3cm]
            \node[main node] (zone) {zone};
            \node[main node] (1) [below left of=zone] {1};
            \node[main node] (2) [below right of=zone] {2};
        %\path [line, green!80!black] (1) -- (zone);
        %\path [line, red] (zone) -- (2);
            \path [line, green!80!black] (1) -- (2);
            %\node (text) [yshift=-5em,above of=zone] {neither};
        %\node at ($(zone) !.5! (2)$) {$\mathbin{\tikz [x=1.4ex,y=1.4ex,line width=.3ex, black] \draw (0,0) -- (1,1) (0,1) -- (1,0);}$};
        \end{tikzpicture}
        \caption{zone as neither origin nor destination node}
    \end{subfigure}
    \caption{zone node and its allowable arc flows}
\end{figure}

\todo[inline]{Maybe a picture of the network explain what the zones are.}

Through out the report,
run-time analysis (big O and other notations) is used to demonstrate the estimation of algorithm running time regarding their input size. 
\todo[inline]{How do I nicely say `let the reader refer to other resources?'
or do I describe what big O notation is?}

\section{Generic Shortest Path Algorithm}
A family of algorithms exists for solving SPP with directed non-negative length edges.
In this section we describe the generic case for these
algorithms, the generic shorest path algorithm (GSP).

This family of algorithms aims at finding a 
vector ($d_1, d_2,\dots d_v$) of distance labels and its corresponding shortest path \citep{Klunder}.
Each $d_v$ keeps the least distance of any path going from $s$ to $v$, $d_v = \infty$ if no path has been found.
A shortest path is optimal when it satisfies the following conditions:
\begin{align}
    d_v \leq d_u + c_{uv}, \quad \forall(u,v) \in E, \label{eq:Bellman1}\\
    d_v  =   d_u + c_{uv}, \quad \forall(u,v) \in P.
\end{align}
\todo[noline]{define $P$}
The inequalities~(\ref{eq:Bellman1}) are called Bellman's condition \citep{Bellman}.
In other words,
we wish to find a label vector $d$ which satisfies Bellman's condition for all of the vertices in the graph.
To maintain the label vector, the algorithm uses a queue $\mathcal{Q}$ to store the label distances.

In the label vector,
a node is said to be unvisited when $d_u = \infty$,
scanned when $d_u \neq \infty$ and is still in the queue,
and labelled when the node has been retrieved from the queue and its distance label cannot be updated further.
If a node is labelled then its distance value is guaranteed to represent the minimal distance from $s$ to $t$.

In the generic shortest path algorithm,
we start by putting the origin node in the queue,
and then iteratively find the arc that violates the Bellman's condition (i.e., $d_v > d_u + c_{uv}$).
Distance labels are set to a value which satisfies condition (\ref{eq:Bellman1}) to the corresponding node of that arc.
Shortest path going from $s$ to all other nodes in $V$ is found when (\ref{eq:Bellman1}) is satisfied for all edges in $E$.
It may not be obvious but negative costs are permitted in the GSP but not negative cost cycles.

We use $p_u$ to denote the predecessor of node $u$.
The shortest path can be constructed by following the predecessor of the destination node $t$ back to the origin node $s$. $p_s$ is often set to $-1$ to indicate it does not have a predecessor.

\todo[inline]{diagram showing $u, v, c_{vw}$ etc.}
\begin{figure}[H]
    \centering
    \tikzstyle{main node} = [circle, draw, text centered, minimum height=2.5em]
    \tikzstyle{line} = [->, draw]
    \begin{tikzpicture}[>=stealth', line width=1pt, auto, node distance=3cm]
        \node [main node] (u)  {u};
        \node [main node] (v) [right of=u] {v};
        \node [main node] (o) [below left of=u, yshift=1cm]  {};
        \node [main node] (d) [below right of=v, yshift=1cm] {};

        \node [above] at (u.north) {$du$};
        \node [above] at (v.north) {$dv = du + c_{vw}$};

        \path [line] (u) -- (v);
        \path [line] (o) -- (u);
        \path [line] (v) -- (d);
    \end{tikzpicture}
    %\caption{zone as neither origin nor destination node}
\end{figure}

Algorithm~\ref{algo:gsp} \citep{Klunder} describes the generic shortest path algorithm mentioned above,
with an extra constraint required when solving a TA problem: travelling through zone nodes is not permitted.
In essence, this algorithm repeatedly selects node $u\in\mathcal{Q}$ and checks the violation of Bellman's condition for all emanating edges of node $u$.
\begin{algorithm}
    \caption{The Generic Shortest Path Algorithm}
    \label{algo:gsp}
    \begin{algorithmic}[1]
        \Procedure{GenericShortestPath}{$s$}
        \State $\mathcal{Q} \gets \mathcal{Q} \cup \{s\}$ \Comment{initialise queue with source node}
        \State $p_s \gets -1$ \Comment{origin has no predecessor}
        \State $d_s \gets 0$
        \ForAll {$ u \in V : u \neq s $} \Comment{all nodes are unvisited except the source}
        \State $d_u \gets \infty$
    \EndFor

    \While{ $\mathcal{Q} \neq \emptyset$ }
    \State $ u \gets \text{next}(\mathcal{Q}) $ \Comment{select next node}
    \State $ \mathcal{Q} \gets \mathcal{Q} \setminus \{u\} $
    \If{ $u \neq \text{zone} $}
    \ForAll {$v : (u, v) \in E$} \Comment{check Bellman's condition for all successors of $u$}
    \If {$d_u + c_{vw} < d_v$}
    \State $d_v \gets d_u + c_{vw}$
    \State $p_v \gets u$
    \If {$v \notin \mathcal{Q}$} 
    \State $\mathcal{Q} \gets \mathcal{Q} \cup \{v\}$ \Comment{add node $v$ to queue if unvisited}
\EndIf
                    \EndIf
                \EndFor
            \EndIf
        \EndWhile
    \EndProcedure
\end{algorithmic}
\end{algorithm}

Algorithm~\ref{algo:gsp} is generic because of two reasons:
the rule for selecting the next node $u$ (the `next' function in line 8) and
the implementation for the queue $\mathcal{Q}$ is unspecified.
Different algorithms use different rules and implementations to give 
either the one-source or the point-to-point shortest path algorithm \citep{mplomer}.
The next two sections describes these rules and implementations.

\section{Label Correcting Algorithm}
\todo[inline]{pseudo code}
\label{section:labelcorrectingalgorithm}
The GSP is addressed as a label correcting algorithm when the queue is a first in first out (FIFO) queue.
\todo{Check if its FIFO or double ended queue}
Given the arc costs can be negative in the GSP,
and in order to satisfy the Bellman's conditions for all edges,
the algorithm has to scan all edges $|V|-1$ times,
giving a run time of $O(|V||E|)$.

In this algorithm,
the distance labels do not get permanently labelled when the next node in the queue is retrieved.
Another node may `correct' this node's distance label again,
thus the name label correcting algorithm.
This algorithm is also called the Bellman–Ford–Moore algorithm credited to \citet{Bellman, Ford} and \citet{Moore}.

\section{Label Setting Algorithm}
\label{section:labelsettingalgorithm}
The classical algorithm for solving the single-source shortest path problem is the label setting algorithm published by \citet{Dijkstra}.
The algorithm is addressed as label setting because when the next node $u$ is retrieved from the queue,
it gets permanently labelled;
the shortest path going to this node is solved and 
the distance label represents the shortest length.
In order to achieve label setting, 
the queue $\mathcal{Q}$ is modified to always have the minimum distance label in front of the queue,
hence the algorithm iterates through every node in the graph exactly once,
labelling the next node $u$ in the order of non-decreasing distance labels.

The advantage of this algorithm over the label correcting algorithm is
that all nodes in the graph are only visited once;
the shortest path tree grows radially outward from the source node.
It is clear that when the next node in the queue is the destination node,
the algorithm can be stopped for the point to point SPP case,
which is desirable for the Path Equilibration method.

\subsection{Priority Queue Implementations}
The run time performance of Dijkstra's algorithm depends heavily on the implementation of the queue for storing the scanned nodes,
\citet{Cormen} suggest the use of a min-priority queues.
Min-priority queues are a collection of data structures that always serve elements with higher priorities. The priority in SPP are the distance labels:
smaller distance label have a higher priority.

Algorithm~\ref{algo:p2pdijkstra} shows the use of the min-priority queue in Dijkstra's algorithm.
The min-priority queue has 3 main operations: Insert, Extract-Min and Decrease-Key.
The Insert operation (line $2$ and $17$ in Algorithm~\ref{algo:p2pdijkstra}) is used for adding new nodes to the queue, the Extract-Min operation (line 8) is used for getting the element with the minimum distance label and the Decrease-Key is used for updating the distance if the node is already in the queue.

\begin{algorithm}[H]
    \caption{Point to Point Dijkstra's Algorithm}
    \label{algo:p2pdijkstra}
    \begin{algorithmic}[1]
        \Procedure{Dijkstra}{$s, t$}
        \State $\text{Insert}(\mathcal{Q}\text{ , u})$ \Comment{initialise priority queue with source node}
        \State $p_s \gets -1$ \Comment{origin has no predecessor}
        \State $d_s \gets 0$
        \ForAll {$ u \in V : u \neq s $} \Comment{all nodes are unvisited except the source}
        \State $d_u \gets \infty$
    \EndFor

    \While{ $\mathcal{Q} \neq \emptyset$ }
    \State $ u \gets \text{Extract-Min}(\mathcal{Q}) $ \Comment{select next node with minimum value}
    \If{ u = t}
    \State $\text{Terminate Procedure}$ \Comment{finish if next node is the destination}
\EndIf
\If{ $u \neq \text{zone} $}
\ForAll {$v : (u, v) \in E$} \Comment{check Bellman's condition for all successors of $u$}
\If {$d_u + c_{vw} < d_v$}
\State $d_v \gets d_u + c_{vw}$
\State $p_v \gets u$
\If {$v \notin \mathcal{Q}$} 
\State $\text{Insert}(\mathcal{Q}, v)$ \Comment{add node $v$ to queue if unvisited}
\Else
\State $\text{Decrease-Key}(\mathcal{Q}, v)$ \Comment{else update value of $v$ in queue}
    \EndIf
\EndIf
                \EndFor
            \EndIf
        \EndWhile
    \EndProcedure
\end{algorithmic}
\end{algorithm}

According to \citet{Cormen},
a min-priority queue can implemented via an array or a binary min-heap, where each implementation give different run time performances.

In the array implementation,
the distance labels are stored in an array where the $n^{\text{th}}$ position gives the distance value for node $n$.
Each Insert and Decrease-Key operation in this implementation takes $O(1)$ time, and each Extract-Min takes $O(|V|)$ time (searching through the entire array), giving a overall time of $O(|V|^2 + |E|)$.

A binary min-heap is a binary tree which satisfies the min-heap property:
the value of each node is smaller or equal to the value of its child nodes.
\citet{Cormen} shows that if the graph is sufficiently sparse (in particular $E = o(|V|^2/\log(|V|))$, Dijkstra's algorithm can be improved with a binary min-heap. In this implementation, the binary tree takes $O(|V|)$ time, Extract-Min takes $O(\log(|V|))$ time for $|V|$ operations and Decrease-Key takes $O(\log(|V|))$ time for each $|E|$. The total running time is therefore $O((|V|+|E|)\log(|V|))$, which improves the array implementation.

The running time can be improved further using a Fibonacci heap
developed by \citet{Fredman}.
Where historically, the development of Fibonacci heaps was motivated by the observation that Dijkstra's algorithm typically makes many more Decrease-Key calls than Extract-Min.
In Fibonacci heap, each of the $|V|$ Extract-Min operations take $O(\log(V))$ amortized time,
and each of the $|E|$ Decrease-Key operations take only $O(1)$ amortized time,
which gives a total running time of $O(V\log(V)+E)$.

Min-priority queue can also be implemented as a binary search tree,
where the worst case for insertion, deletion and search for an element in the tree all run in $O(log(n))$ time.
Dijkstra's algorithm (Algorithm~\ref{algo:p2pdijkstra}) can be modified for a binary search tree implementation: when a label distance of node can be updated, we remove that node from the tree and insert a new one with the update value, which is analogous to the Decrease-Key operation.
Dijkstra's algorithm using a binary search also runs $O((|V|+|E|)\log(|V|))$ in the worst case compared to the min-binary heap.
The advantage of using a binary search tree is that we do not have to keep track of information about whether a node is in the queue,
since we just delete the node from the tree and add the node with a different value, and there is no harm deleting a non-existent node from the tree.

\section{Bidirectional Label Setting Algorithm} \label{section:bidirectional}
Dijkstra's algorithm can be imagined to be searching radially outward like a circle with the origin in the centre and destination on the boundary.
Likewise, Dijkstra's algorithm can be used on the reverse graph (all edges reversed in the graph) from the destination node.
Thus Dijkstra's algorithm can be run on the origin and destination simultaneously at the same time.
The motivation for doing this is because the number of scanned nodes can be reduced when searching bidirectionally:
two smaller circles growing outward radially instead of a larger one.
\todo{illustrate this}

\begin{figure}[H]
    \tikzstyle{main node} = [circle, draw, text centered, minimum height=2.5em]
    \tikzstyle{big circle} = [circle, draw, dashed, text centered, minimum height=8em]
    \tikzstyle{small circle} = [circle, draw, dashed, text centered, minimum height=5em]
    \centering
    \begin{subfigure}[t]{.4\textwidth}
        \centering
        \begin{tikzpicture}[>=stealth', line width=1pt, auto, node distance=2cm]
            \node [main node] (s)  {s};
            \node [main node] (t) [right of=s] {t};

            \node [big circle] at (s) {};
        \end{tikzpicture}
        \caption{Dijkstra's algorithm}
    \end{subfigure}
    \begin{subfigure}[t]{.4\textwidth}
        \centering
        \begin{tikzpicture}[>=stealth', line width=1pt, auto, node distance=2cm]
            \node [main node] (s) {s};
            \node [main node] (t) [right of=s] {t};

            \node [small circle] at (s) {};
            \node [small circle] at (t) {};
            \node [big circle, white] at (s) {};
        \end{tikzpicture}
        \caption{bidirectional}
    \end{subfigure}
    \caption{Difference between the scan area of label setting and its bidirectional version}
\end{figure}

It is common to conclude that the shortest path is found when the two searches meet somewhere in the middle,
but this is not actually the case.
There may exist another arc connecting the two frontiers of the searches that has a shorter path.
\todo{show proof?}
The correct termination criteria was first designed and implementation by \citet{Pohl} based on researches presented by \citet{Dantzig, Nicholson} and \citet{Dreyfus}.
\citet{Klunder} summarises the procedure and algorithm (Algorithm~\ref{algo:bidirectional}) for the termination criteria presented by \citet{Pohl}.

In Algorithm~\ref{algo:bidirectional},
two independent Dijkstra's algorithms are alternatively run on the forward  and reverse graph (forward and backward algorithm),
the algorithms terminate when a node is permanently labelled in both directions.
Once the algorithms have terminated,
the correct shortest path is found by looking for a arc connecting the frontiers of the two searches that may yield a shorter path.
This extra condition increases the run time significantly, 
searches have to be done for all edges that connect all labelled nodes in the forward search to all labelled nodes in the backward search.
\todo{Show theorem and proof?}

Note in Algorithm~\ref{algo:bidirectional},
$\mathcal{R}^s$ is the subset of nodes that are permanently labelled from $s$ with labels $d_v^s$ in the forward search, and 
$\mathcal{R}^t$ is the subset of nodes that are permanently labelled from $s$ with labels $d_v^t$ in the backward search.

\begin{algorithm}[H]
    \caption{Bidirectional Label Setting Algorithm }
    \label{algo:bidirectional}
    \begin{algorithmic}[1]
        \Procedure{Bidirectional}{$s, t$}
        \State Execute one iteration of the forward algorithm.
        If the next node $u$ is labelled permanently by the 
        backward algorithm $(u\in\mathcal{R}^t)$, go to step 3.
        Else, go to step 2.
        \State Execute one iteration of the backward algorithm.
        If the next node $u$ is labelled permanently by the
        forward algorithm $(u\in\mathcal{R}^s)$, go to step 3.
        Else, goto step 1.
        \State Find $\min\{\min\{d_v^s + c_{vw} + d_w^t | v \in \mathcal{R}^s, w \in \mathcal{R}^t, (v, w) \in E\}, d_u^s + d_u^t\}$, which gives the correct shortest path between $s$ and $t$.
    \EndProcedure
\end{algorithmic}
\end{algorithm}

In recent years,
\citet{Goldberg05} improved the bidirectional algorithm using a better termination condition,
where step 3 of Algorithm~\ref{algo:bidirectional} is embed during the searches.
The termination condition is summarized as the following.
During the forward and backward search,
we maintain the length of the shortest path seen so far, $\mu$, and its corresponding path. Initially, $\mu = \infty$.
When an arc $(v,w)$ is scanned by the forward search and $w$ has already been scanned in the reverse search (or vice versa),
we know the shortest $s-v$ and $w-t$ path have lengths $d_v^s$ and $d_w^t$ respectively.
If $\mu > d_v^s + c_{vw} + d_w^t$ then this path is shorter than the one detected before, 
so we update $u$ and its path accordingly.
The algorithm terminates when the search in one direction selects a node already scanned in the other direction.

\citet{GoldbergEPP} showed and proved a stronger termination condition on top of his previous one.
The searches can be stopped if the sum of the top priority queue values is greater than $\mu$:
\[
    \text{top}_f + \text{top}_r \geq \mu
\]
where $\text{top}_f$ and $\text{top}_r$ are the top priority queue values in the forward and reverse search, they the next minimum distance label that have not been labelled.
\todo{Show theorem and proof as well?}

\section{A* Search}
Up until now,
Dijkstra's algorithm does not take into account the location of the destination,
the shortest path tree is grown out radially until the destination is labelled.
In a traditional graph where actual distances are used for the distance labels,
a heuristic can be used to direct the shortest path tree to grow toward the destination (an ellipsoid in stead of a circle).
If the heuristic estimate is the distance from each node to the destination,
and the estimate is smaller than or equal to the actual distance going to that destination,
then a shortest path can be found. This is called A* search or goad directed search, first described by \citet{Astar}.

Formally we define the following.
Let $h_v$ be a heuristic estimate from node $v$ to destination $t$,
and apply Bellman's condition such that an optimal solution exist, that is 
\begin{align}
    &h_v \leq h_u + c_{uv} \quad \forall(u,v) \in E, \\
    &h(t) = 0,
\end{align}
where $t$ is the destination node.
This means the heuristic function $h$ must be admissible and consistent.
The heuristic must never overestimate the actual path length and the estimated cost of a node reaching its destination node must not be greater than the estimated cost of its predecessors.
Note a consistent heuristic is also admissible but not the opposite.
\citet{Astar} proves if the heuristic function (such as using geographical coordinates and Euclidean distance) is admissible and consistent,
then A* is guaranteed to find the correct shortest path with a better time performance by scanning less nodes and edges.

To implement A* search,
Dijkstra's algorithm is modified.
Instead of selecting the node with the minimum distance label in the priority queue,
we select the next node $u$ that has the minimum distance label added with the heuristic value, which is $d_u + h_{ut}$ where $h_{ut}$ is the estimated distance from node $u$ to destination $t$.

In the Path Equilibration method,
geographical coordinates and Euclidean distances can not be used for the heuristic estimate because a travel time function is used for the length of the edges.
Instead, zero-flow travel time from every node to the destination can be used for the heuristic.
Zero-flow travel time admissible and consistent and can be shown by
analysing the travel times function (Figure~\ref{fig:flowfunction}).
The travel times function is a non-decreasing function with the lowest value being the zero-flow travel times.
This means using zero-flow travel times as the heuristic estimate
is assured to be admissible
as no travel time can be lower than the zero flow travel at any time.
The heuristic function is consistent because the travel time from a node to the destination must be no longer than all its predecessors.

\begin{figure}[H]
    \centering
    \begin{tikzpicture}
        \begin{axis}
            [
                domain=0:10000,
                black, no markers, smooth,
                xmin=0,xmax=10000,
                ymin=0,ymax=130,
                axis x line=bottom,
                axis y line=left,
                xlabel={Traffic Flow},
                ylabel={Travel Time},
                every axis y label/.style={at={(current axis.above origin)},anchor=north east, xshift=-2pt},
                every axis x label/.style={at={(current axis.right of origin)},anchor=north west, xshift=-2pt},
                extra y ticks={20},
                extra y tick labels={Freeflow time},
                extra y tick style={overlay},
                xtick=\empty,
                ytick=\empty,
                xticklabel=\empty,
                yticklabel=\empty,
                scaled ticks=false
            ]
            \addplot [->, black] {20+0.15*(x/2000)^4}; 
        \end{axis}
    \end{tikzpicture}
    \caption{Travel time function.}
    \label{fig:flowfunction}
\end{figure}

\begin{comment}
\subsection{Linear Programming Perspective}
\todo[inline]{incomplete}
A* search can be describe in a linear programming perspective.
The reduced cost is $c_{ij} - y_j +y_i$ where $y_j$ and $y_i$ are the dual variables for $(i,j) \in E$, 
essentially A* search puts weight on the edges to and select  out the path with the maximum reduces cost, the weight is the heuristic values.
\begin{alignat}{2}
    \text{Primal: Minimise}   & \quad \sum_{(i, j)\in E} c_{ij} x_{ij} \\
    \text{Subject to} & \quad \sum_j x_{ij} - \sum_j x_{ji} = 0 \quad \forall i\\
    & \quad \sum_j x_{sj} - \sum_j x_{js} = 1 \\
    & \quad \sum_j x_{tj} - \sum_j x_{jt} = -1 \\
    & \quad x_{ij} = \{0,1\} \quad \forall (i,j) \in E
\end{alignat}

\begin{alignat}{2}
    \text{Dual: Maximise} & \quad y_t - y_s \\
    \text{Subject to} & \quad y_j - y_i \leq c_{ij} \quad \forall (i,j) \in E \\
    \text{where} & \quad y_{\cdot} = \sum_j x_{\cdot j} - \sum_j x_{j\cdot}  
\end{alignat}
\end{comment}

\section{Bidirectional A* Search}
Bidirectional search can also be applied to A* search,
where two ellipsoids are extended from the origin and destination respectively.
One may construct the shortest path with the same termination condition described in section~\ref{section:bidirectional}.
But this would not work.
This is due to fact that A* search does not label the nodes permanently in the order of their distance from the origin \citep{Klunder},
the heuristic estimations are no longer consistent.

The strategy for the correct use of heuristic estimates and termination criterion has first been published by \citet{Pohl}. The use of heuristic estimates is later improvement by \citet{Ikeda} and the termination criterion is improved by \citet{GoldbergEPP}.

The strategy is as follows. The heuristic estimates need to translated to consistent functions first. 
We denote $\pi_f(v)$ the estimate on distance from node $v$ to the destination $t$ in the forward search and $\pi_r(v)$ the estimate on distance from origin $s$ to node $v$ in the backward (reverse) search.
\todo{illustrate!}
\begin{figure}[H]
    \tikzstyle{main node} = [circle, draw, text centered, minimum height=2.5em]
    \tikzstyle{big circle} = [ellipse, draw, dashed, text centered, minimum width=8em, minimum height=5em]
    \tikzstyle{small circle} = [ellipse, draw, dashed, text centered, minimum width=5.5em, minimum height=3.5em]
    \centering
    \begin{subfigure}[t]{.4\textwidth}
        \centering
        \begin{tikzpicture}[>=stealth', line width=1pt, auto, node distance=3cm]
            \node [main node] (s)  {s};
            \node [main node] (t) [right of=s] {t};

            \node [big circle] at ($(s)+(0.9,0)$) {};
        \end{tikzpicture}
        \caption{A* search}
    \end{subfigure}
    \begin{subfigure}[t]{.4\textwidth}
        \centering
        \begin{tikzpicture}[>=stealth', line width=1pt, auto, node distance=3cm]
            \node [main node] (s) {s};
            \node [main node] (t) [right of=s] {t};

            \node [small circle] at ($(s)+(0.4,0)$) {};
            \node [small circle] at ($(t)-(0.4,0)$) {};
            \node [big circle, white] at (s) {};
        \end{tikzpicture}
        \caption{bidirectional}
    \end{subfigure}
    \caption{Difference between the scan area of A* search and its bidirectional version}
\end{figure}
In general two arbitrary feasible functions $\pi_f$ and $\pi_r$ are not consistent, but their average is both feasible and consistent \citep{Ikeda}:
\begin{align}
    p_f(v) = \frac{1}{2}(\pi_f(v)-\pi_r(v)) + \frac{\pi_r(t)}{2} \\
    p_r(v) = \frac{1}{2}(\pi_r(v)-\pi_f(v)) + \frac{\pi_f(s)}{2} 
\end{align}
where the two constants $\frac{\pi_r(t)}{2}$ and $\frac{\pi_f(s)}{2}$ are added by \citet{GoldbergEPP} to provide better estimates.
Note the modified consistent heuristic $p$ provides worse bounds than the original $\pi$ values.

Finally \citet{GoldbergEPP} shows and proves the stopping criterion:
\begin{align}
    \text{top}_f + \text{top}_r \geq \mu + p_r(t),
\end{align}
where is $\mu$ the best $s-t$ path seen fast, $\text{top}_f$ the length of the path from $s$ to the top node (minimum distance label) in the forward search priority queue and $\text{top}_r$ the length of the path from $t$ to the top node in the backward search priority queue.

\section{Preprocessing and More}
\todo[inline]{this section is in draft stage}
Preprocessing - trade memory to get faster time.
We can either do a fast preprocessing between iterations to make query in each iteration (so combined speed is still faster) 
or do a long preprocessing at the start and use the computed heuristic values
\begin{itemize}
    \item A* landmarks and triangle inequality (ALT)
    \item Reach-based routing 
    \item ALT + Reach
    \item Geometric Containers
    \item Arc Flags
\end{itemize}

If we have more data on the network we can use
algorithms that use hierarchies.
Consider roads with higher speed first: use a hierarchy of subgraphs.
\begin{itemize}
    \item Radius search.
    \item multi-level approach
    \item highway hierarchies
\end{itemize}

\todo[inline]{Extract from: Speed-Up Techniques for Shortest-Path Computations by Dorothea Wagner, Thomas Willhalm, and Fast Shortest Path Algorithms for Large Road Networks by Faramroze Engineer }

We can also try Lifelong Planning A* (LPA*), using heuristic from previous each iteration, but the original paper says only a few percent arc change can boost run time, not idea if it is more than that. 

If the edge lengths are whole numbers then we can use multi-level bucket for the priority queue.


