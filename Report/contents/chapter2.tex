\chapter{Problem Description}
In the previous chapter we have briefly talked about the
shortest path and the traffic assignment problem,
In this chapter, we give formal definition to these concepts,
as well as the idea of an data structure.
The network problem data that is going to be tested on is also 
described in this chapter.
\section{Notations and Definitions}

In the context of transportation networks,
we use the notation $ \mathcal{G} = ( \mathcal{V}, \mathcal{A} ) $ for a directed graph,
where $ \mathcal{V} $ denotes the set of nodes (origins, destinations, and intersections)
and $ \mathcal{A} $ the set of arcs (roads);
so $ \mathcal{A} $ is a subset of the set $ \{ (u, v)\, | \, u, v \in \mathcal{V} \} $ of all ordered pairs of nodes.
We denote the cardinality of $ \mathcal{V} $ be V and $ \mathcal{A} $ be A.
We assume that $ 1 \leq V < \infty $ and $ 0 \leq A < \infty $,
and that a function $ c : \mathcal{A} \rightarrow \mathbb{R} $ is given that assigns a cost (travel time) to any arc $ (u,v) \in \mathcal{A} $.
We write the costs of arc $(u, v)$ as: $ c((u, v)) = c_{uv} $.

The path inside a transportation network has to be a directed simple path, 
which is a sequence of nodes and arcs $ (u_1, (u_1, u_2), u_2, \cdots , (u_{k-1}, u_k), u_k ) $
such that $ (u_i, u_{i+1}) \in \mathcal{A}$ for $i = 1,\cdots,k-1$ and $u_i \neq u_j$ for all $ 1 \leq i < j \leq k$.
Note $u_1$ is the origin and $u_k$ is the destination of the path $P$ called an O-D pair;
in a transportation network,
these O-D pairs are often traffic zones for generating supplies and receiving demands,
this means the nodes for these O-D pairs are not travelable,
traffic flows cannot go through these zone nodes.
\marginpar{TODO\\diagram\\explain}
Finally we denote cost of the whole path $C(P) := \sum_{(u,v)\in P} c_{vw}$.

\subsubsection{Shortest Path Problem}
The Shortest Path Problem (SPP) is the problem of finding the distance for a given origin $s$ (source) and a destination $t$ (target).
We assume the graph contains a path from node $s$  to node $t$, as well as all arc lengths are positive.
For a real transportation network, all these assumptions are satisfied naturally; any transportation networks will have at least one O-D pair and
all arc lengths are travel times that are naturally positive.
\marginpar{SPP}

{
    Adapted from (How do I reference for all these paragraphs?)
    \begin{verbatim}
    The Shortest Path Problem on Large-Scale Real-Road Networks
    NETWORKS—2006—DOI 10.1002/net
    \end{verbatim}
}
There are two types of SPP that are going to
be analysed in this report,
a single-source SPP and a point to point SPP.
A single-source SPP solves the shortest path going from one origin to all other destinations in the network,
meanwhile a point to point SPP solves from one origin to a specific destination.

\subsubsection{Traffic Assignment}
Traffic Assignment (TA) is the problem of selecting paths between origins and destinations in a transportation network,
and identify how many travellers use each path.
The difficulties of TA lies within a realistic model of the travel times,
travel times are modelled by a non-linear function to capture congestion effects:
more traffic flow means more congestion leading to slower travel times.
\marginpar{TA}

\subsubsection{Path Equilibration}
\marginpar{mention\\user/system\\optimal?}
Path Equilibration (PE) is a method for solving the TA problem.
This method assumes all travellers are selfish and will always find the shortest (least travel time) path.
Initially all travel demand is assigned to the shortest path between each O-D pair based on zero-flow travel times.
Then the travel times are updated iteratively based on the new flows, 
assigning new shortest path and new flows in each iteration,
eventually the travel times will reach equilibrium and no better shortest path can be calculated,
resulting optimality.
For each iteration of PE,
the point to point SPP is solved for all O-D pairs in the network.

\marginpar{PE}

\subsection{Forward Star}\label{chap:forwardstar}
Data structure for storing the network,
useful for label correcting algorithm.

\subsection{Priority Queue}
A priority queue is a data structure which sorts elements by its priority,
element with high priority is always retrieved first before an element with lower priority.
Useful for label setting algorithm.

\section{Problem Data and Result Explanation}
Data used for solving the TA problems are retrieved from (http://www.bgu.ac.il/~bargera/tntp/),

Through the report the following Table is used to show the 
run time and number of iterations took.
O-D pairs are the pair of origin and destination pair for the zone nodes in the network,
zones are sometimes only a origin or a destination,
thus we 
Run time (seconds) is measured from executing the 
path equilibration algorithm.
Iterations is how many times the whole network
get solved to settle the traffic flows to equilibrium.
\begin{table}[H]
    \centering
    \begin{tabular}{ccccccc}
        Network & Nodes & Zones & O-D pairs & Arcs & Run Time (s) & Iterations \\
        SiouxFalls    & 24   & 24  & 528   & 76   \\
        Anaheim       & 416  & 38  & 1406  & 914  \\
        Barcelona     & 1020 & 110 & 7922  & 2522 \\
        Winnipeg      & 1052 & 147 & 4344  & 2836 \\
        ChicagoSketch & 933  & 387 & 93135 & 2950 
    \end{tabular}
    \caption{Network Problem Data}
\end{table}
The nodes the table include the traffic zones.
By examining the network problem data,
we can see that the number of O-D pairs increase
significantly respect to the number of zone nodes,
this is important because it indicates how many SPPs need to be solved for each iteration of the PE.
We can also roughly tell that these networks are very sparse,
as a complete graph (every node is connected to every other node) of 1000 nodes have 499500 arcs ($n(n-1)/2$),
and the larger networks in our problem data only have about 0.4\% to 0.6\% of arcs in a complete graph, this information is useful
when we start tuning the algorithms for solving SPP.

Most of the data does not resemble a real world transportation network, 
for example all roads have the same speed limit, road type and capacity.

All results are from an Intel i5 1.78ghz CPU computer, 4GB RAM, running Ubuntu 12.04 Linux.
Code is compile with g++ with -O3 optimisation flag.

The accuracy of all results are checked by comparing the traffic flows from the traffic assignment output,
as well as the final shortest path for every O-D pairs.
\marginpar{CPU\\param}

