\chapter{Implementation Details} \label{chap:implementation}

The previous chapters have described all the algorithms that are going to be considered for this report. 
In this chapter, specific implementation details of the algorithms that provide better performance are discussed.

\section{Traffic Assignment Implementation}
The path equilibration algorithm and other algorithms for solving the traffic assignment problem has already been implemented by Olga Perederieieva, the co-co-supervisor of this project.
The algorithms are implemented in the C++ programming language,
where the language has a superior run time performance compared to the others.

The current implementation of the path equilibration algorithm uses the Bellman-Ford-More algorithm for its point-to-point shortest path calculations.
When solving the traffic assignment problem,
the algorithm will spend most of its time computing shortest paths.
The algorithm will also spend time running parts of the algorithm,
but most of that time is dedicated to the convergence check step mention in Section~\ref{sec:convergence} ,
where the algorithm require to run the Bellman-Ford-More algorithm on all of the zones for the all-or-nothing solution.

\section{Graph Storage}
The storage of the underlying graph need to be implemented in such a way that it can provide the most efficient access to its nodes and arcs.
The most efficient implementation is the Forward Star data structure described in \citet{Sheffi}.
The data structure compactly stores graphs in $O(|N|+|A|)$ spaces,
and provides $O(1)$ random access to all of its nodes,
as well as $O(1)$ access to all emanating arcs of a chosen node.
Using the Forward Star ensures that run time of accessing the graph can be neglected when analysing shortest path algorithms.

\section{Priority Queue Implementations} \label{sec:pq_implementation}
As mentioned in the Dijkstra's algorithm section (Section~\ref{sec:dijkstra},
its performance is heavily depended on the implementation of the priority queue.
Various priority queue implementations exist,
they include:
\begin{itemize}
    \item $\langle$priority\_queue$\rangle$ from the C++ standard template library - an array based heap implementation,
    \item $\langle$set$\rangle$ from the C++ standard template library - a binary search tree implementation,
    \item $\langle$heap$\rangle$ from the C++ Boost library - a family of pointer based heap implementations.
\end{itemize}
Each priority queue implementation has some advantages than the other,
for example some provide faster tree balancing while others provide faster Extract-Min or Delete operation.
For this project, all of these implementations are going to be experimented.

First we examine the 6 variants of Heap implementations from the C++ Boost $\langle$heap$\rangle$ library, shown in Table~\ref{table:heaps}.
In the table, N is the number of elements in the priority queue and
the time complexities are measured in amortized time\footnote{
Amortized time: how much time is taken in total when an operation is repeated a millions for example, with different inputs. Run time is averaged out between the worst-case and the best-case.}.

\begin{table}[H]
    \centering
    \begin{tabular}{cccccc}
        & top() & push()    & pop()     & increase() & decrease() \\
        d-ary (Binary)  & O(1)  & O(log(N)) & O(log(N)) & O(log(N))  & O(log(N))  \\
        d-ary (Ternary) & O(1)  & O(log(N)) & O(log(N)) & O(log(N))  & O(log(N))  \\
        Binomial        & O(1)  & O(log(N)) & O(log(N)) & O(log(N))  & O(log(N))  \\
        Fibonacci       & O(1)  & O(1)      & O(log(N)) & O(1)       & O(log(N))  \\
        Pairing         & O(1)  & O($2^{2*\log(\log(N))}$) & O(log(N)) & O($2^{2*\log(\log(N))}$) & O($2^{2*\log(\log(N))}$) \\
        Skew            & O(1)  & O(log(N)) & O(log(N)) & O(log(N)) & O(log(N))   
    \end{tabular}
    \caption{C++ Boost Heap Implementations with Comparison of Amortized Complexity \citep{BoostHeap}}
    \label{table:heaps}
\end{table}

We are interested in using Boost library Heaps rather than the C++ standard library Heap is due to one reason:
the decrease (or increase) function.
The decrease (or increase) function is referred as the decrease-key (or increase-key) operation mentioned in the priority queue section (Section~\ref{sec:dijkstra}).
The difference between decrease and increase is that Decrease-key is used for a min-heap trees (minimum value on top) and increase-key is used for max-heap trees (maximum value on top).
The function is used to change the distance labels in the priority queue when a node is scanned and updated during the shortest path calculations.
In Dijkstra's algorithm,
often nodes are scanned multiple times in the label updating step,
so instead of adding the node again into the Heap tree with a difference value,
we can just use the decrease-key operation.
The advantage of using this operation is that we can reduce the size of the Heap tree, which results a performance improvement as it takes less time to search and add nodes to small heap trees.

In table~\ref{table:heaps},
we observe the Fibonacci Heap has a very interesting time complexity;
it has a constant amortized time ($O(1)$) for the push, pop and increase-key operation.
This is very attractive for us, but the problem is that
we do not know how much constant time it really uses behind its big O notation.

C++ Boost Library Heaps are implemented as max-heaps,
which means in order to use the Fibonacci O(1) increase-key function,
all of the distance labels need to be negated when added to the heap.

Next we examine $\langle$priority\_queue$\rangle$ from the C++ standard library.
This implementation also provides $O(\log(N))$ push and pop operation,
but it does not have the decrease-key operation nor does it have a way to change node values somewhere else in the tree.
So when solving the shortest path problem,
the priority queue is going to have many nodes that are the same but with different distance labels.
This is not a problem for our shortest path algorithms.
When a node is added to the queue more than once with different distance labels,
the one with the smaller distance label is always going to be in front of the queue waiting to be labelled first,
so once that node is labelled,
all the other same node will simply be ignored in the algorithm.

Finally we examine $\langle$set$\rangle$ from the C++ standard library.
A set is a data structure used to store unique elements that follow a specific order.
In the C++ standard library, it is implemented as a red-black binary search tree.
This data structure can be used for our shortest path algorithms because it provides $O(\log(N))$ insert, search and delete operations.
For our shortest path algorithms,
we can modify them to accommodate the unique elements and specific ordering requirement.
To meet the unique elements requirement,
instead of using the Decrease-key operation whenever a node need to be updated,
we simply delete that node and insert the one with the new value.
And for the ordering requirement,
we can just order the nodes non-decreasingly by their distance labels,
so the node with the minimum label always come first.
The advantage of $\langle$set$\rangle$ compared to $\langle$priority\_queue$\rangle$ is that nodes can be removed from anywhere in the data structure,
so $\langle$set$\rangle$ can be faster than $\langle$priority\_queue$\rangle$ because the performance of their operations are heavily dependent by the number of nodes in the data structures.
