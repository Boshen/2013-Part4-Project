\chapter{Introduction}
\section{Introduction to traffic modelling}
Nowadays a large portion of people's daily lives involve activities that relate to transportation.
For example, most people need to travel between their workplace and residence twice a day,
or buy goods from shops that need to be delivered across the city.
In the meanwhile, 
transportation networks expand and improve constantly to cater for people's demand for efficient travelling.
but the rate of improvement does not confront with the rate of population growth.
As a result,
the network becomes inefficient, resulting in traffic congestion.

Congestion leads to major economical losses due to time delays and increased usage of petrol.
Congestion causes air pollution that increases respiratory problems such as asthma, 
and the exhaust gas exacerbates global warming.
Congestion also increases noise pollution and causes frustration,
which in turn accelerates traffic accidents.
It is important for traffic designers to be able to reduce congestion problems,
and eliminate its negative effects.
%The transportation network can be improved by, for example,
%introducing road tolls to diverge traffic to less congested roads,
%or by educating people to use public transport.

Making improvements to the transportation network tends to be very costly,
so a good plan is necessary:
use the least amount of investment for the greatest change.
In order to make better plans for traffic design,
different mathematical models have been built to estimate the current and future behaviour of the transportation system.
One particular model called the transportation forecasting model is commonly used.
The aim of this model is to estimate future traffic usage when the system is changed.
For example, changes to a transport system can include upgrading or adding roads, changing roundabouts to traffic lights, or providing better public transport. 

The transportation forecasting model has 4 stages (shown in Figure~\ref{fig:model}):
trip generation, trip distribution, mode choice and traffic assignment \citep{Sheffi}.
In the model, the solution of each stage can be passed to the previous ones to improve transportation forecasting and traffic design.
This means the model can be calibrated numerous times to establish an accurate model which can estimate future traffic behaviour.
In summary, 
the model collects traffic demand data and 
generates origins and destinations for travellers (trip generation).
It then calculates the number of trips that are required between each origin and destination (trip distribution),
and decides which transportation method should be used for each trip (mode choice).
Finally it decides the route (e.g.\ the shortest path) that each trip needs to travel on (traffic assignment), 
where traffic flow is estimated for every modelled road in the network.
The traffic assignment problem in the last stage of the forecast model is a complicated problem.
This is because congestion occurs as traffic flows are assigned onto the network,
and it is very difficult to find an equilibrium situation where everybody in the network finds their best route.
Generally the equilibrium situation happens when it is assumed that people are selfish,
everyone tries to minimise travel times by travelling along their shortest paths.

\begin{figure}[!ht]
    \centering
    \tikzstyle{block} = [rectangle, draw, text width=10em, text centered, rounded corners, minimum height=2em]
    \tikzstyle{line} = [draw, -latex']
    \begin{tikzpicture}[node distance=4em]
        \node [block] (first) {Trip Generation};
        \node [block, below of=first] (second) {Trip Distribution};
        \node [block, below of=second] (third) {Mode Choice};
        \node [block, below of=third] (fourth) {Traffic Assignment};
        \path [line] (second) -- (third);
        \path [line] (first) -- (second);
        \path [line] (third) -- (fourth);
        \path [line] (fourth.west) -- ($(fourth.west)-(0.8,0)$) -- ($(first.west)-(0.8,0)$) -- (first.west);
        \path [line] ($(second.west)-(0.8,0)$) -- (second.west);
        \path [line] ($(third.west)-(0.8,0)$) -- (third.west);
    \end{tikzpicture}
    \caption{Transportation forecasting model}
    \label{fig:model}
\end{figure}

\section{Purpose of this project}
The transportation forecasting model has been implemented in many software packages for traffic design.
One key observation is that,
the traffic assignment problem can take hours, or even days to solve.
It turns out that,
the bottleneck is in the algorithm for solving the shortest path problem.
This is because algorithms for solving the traffic assignment problem are usually iterative,
where each iteration requires to find millions of shortest paths between every origin and destination in the network.
Although each shortest path calculation may take only a fraction of a second,
cumulatively the computational time can be very long.
Therefore the purpose of this project is to find a faster algorithm for solving the shortest path problem in an iterative environment.
As a result, traffic assignment will be solved faster
for larger and more complicated road networks.
This will also allow city designers to estimate traffic flows further into the future and make better decisions on road network design.

In summary,
The algorithms that are going to be studied are:
\begin{itemize}
    \item Bellman-Ford's label correcting algorithm,
    \item Dijkstra's label setting algorithm (using different data structures),
    \item Bidirectional Dijkstra's algorithm,
    \item A* search,
    \item Bidirectional A* search.
\end{itemize}

We also aim to find and discuss the following techniques that can speed up shortest path calculations in an iterative environment:
\begin{itemize}
    \item preprocessing algorithms,
    \item use information from the previous iterations,
    \item avoid shortest path calculations.
\end{itemize}

\section{Structure of the report}
Chapter~\ref{chap:ta} gives a short description of the traffic assignment problem
and presents a specific algorithm called the path equilibration algorithm that solves it.
This algorithm is the begin of our experiments for this project.
Chapter~\ref{chap:solvingspp} introduces the shortest path problem and presents some of the well established algorithms that solve it.
Implementation details are presented in Chapter~\ref{chap:implementation}.
%Strategies for solving the shortest path problem faster in an iterative environment are also described in this chapter.
Chapter~\ref{chap:results} shows the results and discussions for the algorithms described in Chapter~\ref{chap:solvingspp}.
Chapter~\ref{chap:iterative} presents strategies and results for solving the shortest path problem in path equilibration algorithm of the traffic assignment.
Finally conclusions and future work are drawn in Chapter~\ref{chap:conclusions}.
