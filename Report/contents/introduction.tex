\chapter{Introduction}
{
    SPP has been mentioned countless times,
    most materials are very repetitive,
    what kind of info do i need to make a reference of?
}
\section{Traffic Assignment Models}

In today's world, cities are becoming larger, 
road networks are becoming more complex,
and city designers are facing difficulties of 
forecasting, building and resigning the roads.
Road networks often face congestions that are hard to predict,
leading to unpredictable travel times for people to travel from places to places.

In order to solve these problems,
different traffic models are built.
One particular model is called the transportation forecasting model.
\marginpar{forecast\\ model}
This model solves traffic flows in a typical road network,
of which involves four stages of process: trip generation,
trip distribution, mode choice and traffic assignment.
\marginpar{4 stage}
In short,
this model generates traffic demands and supplies in different traffic zones,
\marginpar{zones}
having them act as origins and destinations for the travellers.
\marginpar{O-D pair}
this model then selects a transportation method for travellers
to use, and sends them to their destination.
This final stage of sending travellers to their destination is
addressed as the traffic assignment (TA) problem.
\marginpar{TA}
TA is a complex task,
which involves the selection of routes to use by considering information such as congestions.

One method of solving the traffic assignment is called 
equilibrium assignment or the path equilibration method.
\marginpar{PE}
In this method, 
shortest path for every origin-destination pair (O-D pair) are calculated, 
\marginpar{Shortest\\Path}
and traffic flows are assigned to theses shortest path.
The traffic assignment is said to be solved when equilibrium occur when no traveller can find a shorter path.
As every time traffic flows are reassigned,
congestion will happen differently and 
travel time for every path will change respectively,
thus a number of iterations is needed to settle the traffic flows
to equilibrium.
The travel times for the path are no longer expressed solely by their
distance,
but instead they are expressed by a formulation which considers
parameters such as the travel time from the previous iteration,
the number of travellers and the capacity of the road.
\marginpar{travel\\time}

Transportation networks are typically very large,
which means it may take a very large numbers of iterations
for the traffic flows to settle,
and meanwhile in each iteration,
shortest path for each O-D pair are calculated repeatedly,
which is a very time consuming step for large networks with lots of roads and intersections.
\marginpar{time\\consuming}
Speeding up the shortest path calculation would significantly speed up the TA algorithms,
when everything is sped up,
larger and more complex networks can then be solved faster.
And when this fast TA algorithm is put back inside the traffic forecasting model,
we can predict longer into the future by changing traffic demands and supplies or modifying the road network design.

Existing computer code (written in C++) already exist for the TA algorithms,
including the path equilibrium method and many others.
\marginpar{existing\\code}
A simple shortest path calculation algorithm is currently used by the path equilibrium method,
which runs very poorly given a small network.

The existing algorithm is referred as the label-correcting algorithm, 
it is modified and implemented as the well known label-setting Dijkstra's algorithm using a special data structure for storing information,
A* algorithm is them implemented to speed the Dijkstra's algorithm for the point to point shortest problem.

\marginpar{zone\\no\\travel}
Special cares are taken when implementing these algorithms,
namely the origins and destinations mentioned sometimes do not act as intersections for the travellers (to be included in the shortest path calculation),
since these origins and destinations are conceptual traffic zones for generating demands and supplies.
