\chapter{Introduction}

\todoin[inline]{CHECK \LaTeX LOG!!}
\todoin[inline]{use italics and abbreviations}

\section{Introduction to traffic modelling}
Nowadays a large portion of people's daily lives involve activities which relate to transportation,
For example, most people need to travel between their workplace and residence twice a day,
or buy goods from shops where they need to be delivered across the city.
In the meanwhile, 
transportation networks expand and improve constantly to cater people's demand for an efficient transportation network,
but the rate of improvement does not confront with the rate of population growth.
As a result,
the network becomes inefficient, causing traffic congestion.

Congestion lead to major economical losses due to time delays and increase usage of petrol.
Congestion cause air pollutions that increase respiratory problems such as asthma, 
And the exhaust gas exacerbates global warming.
Congestion also increase noise pollution and cause frustration,
which in turn accelerate traffic accidents.
It is important for traffic designers to be able to reduce congestion problems,
and eliminate its negative effects.
The transportation network can be improved by for example,
introduce road tolls to diverge traffic to less congested roads,
or educate people to use public transports.

Making improvements to the transportation network tend to be very costly,
so an optimal plan is always necessary:
use the least amount of investment for the greatest change.
In order to make optimal plans for traffic design,
different mathematical models have been built to simulate the current and future behaviour of the transportation system.
One particular model called the transportation forecasting model is commonly used.
The aim of this model is to estimate future traffic usage when the system is changed.
For example, upgrading or adding roads, changing roundabouts to traffic lights or provide better public transports. 

The transportation forecasting model has 4 stages (shown in Figure~\ref{fig:model}):
trip generation, trip distribution, mode choice and traffic assignment.
In the model,
each of the next stage can pass information to the previous stages to improve traffic design.
In summary, 
the model collects traffic demand data and 
generates origins and destinations for travellers(trip generation),
it then calculates the number of trips that are required between each origin and destination (trip distribution),
and decides which transportation method should be used for each trip(mode choice),
finally it decides the best route (e.g.\ shortest path) that each trip need to travel on (traffic assignment).
The traffic assignment problem in the last stage of the forecast model is a very complicated problem.
This is because congestion occur as traffic flows are assigned onto the network,
and it is very difficult to find an equilibrium situation where everybody in the network find their best route.

\begin{figure}[H]
    \centering
    \tikzstyle{block} = [rectangle, draw, text width=10em, text centered, rounded corners, minimum height=2em]
    \tikzstyle{line} = [draw, -latex']
    \begin{tikzpicture}[node distance=4em]
        \node [block] (first) {Trip Generation};
        \node [block, below of=first] (second) {Trip Distribution};
        \node [block, below of=second] (third) {Mode Choice};
        \node [block, below of=third] (fourth) {Traffic Assignment};
        \path [line] (second) -- (third);
        \path [line] (first) -- (second);
        \path [line] (third) -- (fourth);
        \path [line] (fourth.west) -- ($(fourth.west)-(0.8,0)$) -- ($(first.west)-(0.8,0)$) -- (first.west);
        \path [line] ($(second.west)-(0.8,0)$) -- (second.west);
        \path [line] ($(third.west)-(0.8,0)$) -- (third.west);
    \end{tikzpicture}
    \caption{Transportation forecasting model}
    \label{fig:model}
\end{figure}



\begin{comment}
A traffic model called the transportation forecasting model is built
with the aim to reduce congestion and predict future traffic response when the behaviour of the traffic is changed.
This model solves and estimates traffic flows for a given time period with the following four stages: trip generation,
trip distribution, mode choice and traffic assignment (Figure~\ref{fig:model}).
In short,
this model generates origins and destinations for travellers to travel from and to in different parts of the road network,
it then calculates the number of trips that are required for each origin and destination pair
and computes the proportion of trips between each pair that use a particular transportation method,
in the end it assumes all travellers choose the best trip with the least transportation cost and best transportation method (e.g.\ shortest path, least travel time or cheapest route) and
assigns each traveller to their destination considering traffic congestion.

The traffic assignment (TA) problem in the last stage of the forecast model is a very complicated problem, 
this is because the problem is only solved when the network reaches user equilibrium,
which means no traveller can lower their transportation cost through unilateral action: every traveller will strive to find the shortest path while ignoring all other travellers.
\todo[inline]{reference user equilibrium - John Glen Wardrop principles of equilibrium}
User equilibrium is difficult to find because in traffic assignment,
travel times on different roads are modelled as nonlinear functions to capture congestion effects (more traffic flow results slower travel time).
As different routes are assigned to the travellers,
congestion happen differently for each road in a nonlinear manner,
making the result of relocation of travellers hard to calculate.

One method of solving the traffic problem is the Path Equilibration (PE) method \citep{Florian}.
This method initially calculates the shortest paths between each trip origin and destination based on the zero-flow travel times.
Traffic flows are assigned to these shortest paths and new travel times are updated accordingly.
The method iteratively re-identifies new shortest path based on the new travel times and re-assigns traffic flows until user equilibration is reached.

Both of these methods are iterative methods that require shortest path calculations for every trip origins in the network.
It is not difficult to imagine that there would be many shortest path calculations if the network has hundreds of origins and destinations and takes some iterations to solve.
Each shortest path calculation would also be very hard to solve if the network has a few hundred intersections and a few thousand roads for a realistic city road network.
\citet{Sheffi} states that finding the shortest path is the most computation intensive component for the PE or Frank-Wolfe algorithms, 
other components in the algorithms such as updating new values and convergence check only requires a few percentages of the total running time.
Overall, speeding up shortest path calculations would significantly speed up the traffic assignment algorithms.
As a result,
traffic forecasting would be solved faster for larger and more complicated road networks,
which allow city designers predict traffic further into the future and make better decisions on road network design.
\end{comment}

\section{Purpose of this project}
The transportation forecasting model has been implemented in many software for traffic design.
One key observation from these software is that,
the traffic assignment problem takes days, or even weeks to solve.
It turns out that,
the bottleneck is in the algorithm for solving the shortest path problem.
This is because algorithms for solving the traffic assignment problem are usually iterative,
where each iteration (sometimes there are hundreds of iterations) require to find millions of shortest paths between every origin and destination in the network.
Although each shortest path calculation may take only a fraction of the time,
but cumulatively the computation is huge.
So the purpose of this project is to find a faster algorithm for solving the shortest path problem in an iterative environment.
As a result, the traffic assignment will be solved faster
for larger and more complicated road networks,
and this will allow city designers estimate traffic flows further into the future and make better decisions on road network design.

\begin{comment}
The algorithm that are going to be tested are:
\begin{itemize}
    \item Bellman-Ford Label Correcting Algorithm,
    \item Dijkstra Label Setting Algorithm (using different data structures),
    \item Bidirectional Dijkstra,
    \item A* Search,
    \item Bidirectional A* search.
\end{itemize}

This project also aims to find and discuss techniques that can speed shortest path calculations in an iterative environment:
\begin{itemize}
    \item network preprocessing,
    \item using information from the previous iteration for the current iteration to avoid recalculating shortest paths that are not going to change.
\end{itemize}
\end{comment}

\section{Structure of the report}
Chapter~\ref{chap:ta} gives a short description of the traffic assignment problem
and presents a specific algorithm (The Path Equilibration Algorithm) that solves it.
This algorithm is going to be experimented for this project.
Chapter~\ref{chap:solvingspp} introduces the shortest path problem and presents some of the well established algorithms that solves it very fast.
Strategies for solving the shortest path problem faster in an iteration environment is also describe in this chapter.
Implementation details are presented in Chapter~\ref{chap:implementation}.
And Chapter~\ref{chap:results} shows the results and their discussions.
Finally conclusions are drawn in Chapter~\ref{chap:conclusions}.
