\section*{Abstract}
\begin{comment}
One of the major concerns of commuters in Auckland (and all over the world!) is road traffic
and traffic congestion. Both the design of the road network and determining the best way to
use it are complex questions that can be addressed based on so-called traffic assignment (TA)
models, which are network equilibrium problems. These models can be used to assess the
benefit of road upgrades or adding new roads to the network.
A TA problem models the route choice of users of a network with nonlinear travel time
functions to capture congestion effects as more traffic flow on a road means slower travel time.
The aim of TA is to identify how many network users end up choosing to travel along each
individual arc in the network. This is done by assuming network users are selfish and choose to
travel along their shortest path between origin and destination.
In solving such a TA equilibrium problem, an iterative approach is used. Initially all travel
demand is assigned to the shortest paths computed for between each trip origin and destination
based on zero-flow travel times. Then, travel times are updated based on the new flows. New
shortest paths are identified and travel demand is re-assigned. This process continues until it
eventually reaches the equilibrium solution.
The iterative scheme used by TA algorithms requires many shortest path calculations.
Speeding up these shortest path calculations can lead to a significant speed up of TA
algorithms, but the speed up technique used would differ for particular traffic assignment
algorithms. One idea is to use A* search for TA algorithms that repeatedly compute shortest
paths between single origin destination pairs. A* should lead to a significant improvement in
runtime over using a standard label setting algorithm. Other speed-up techniques for shortest
path algorithms will be identified and tested, exploiting the fact that while shortest paths
change in every iteration of the algorithm, it may be possible to avoid fully re-computing
shortest path trees. Avoiding re-computation of shortest path trees should benefit TA
algorithms that require shortest paths between a single origin and all destinations.
\end{comment}
