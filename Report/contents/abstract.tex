\begin{center}
\Huge Abstract
\end{center}
\vspace{1cm}
The Traffic Assignment (TA) problem involves the selection the optimal path for every vehicle in a transportation network.
One algorithm for solving TA is the Path Equilibration (PE) algorithm.
PE requires solving shortest path repeatedly between every origin and destination pair in the network for a large number of iterations, while the edge costs change between iterations.
The aim of this project is to find a faster shortest path algorithm for PE.
We implement Dijkstra's algorithm and A* search, their bidirectional versions and 8 different versions of priority queue data structures that improve these algorithms' performance.
We develop two strategies for using these algorithms in the iterative environment of PE.
The first strategy is to avoid next few numbers of iterations when the shortest path of the previous two iterations is the same.
The second strategy is to randomly skip the next shortest path calculation, where we hope for the situation where the previous and current iteration is going to be the same.
We present experimental results that demonstrate the run time differences between these priority queues, algorithms and strategies and show that A* search algorithm with random skipping strategy has the best performance.
