\chapter{Summary and Conclusions} \label{chap:conclusions}
To summarise,
in this project we have studied the point-to-point shortest path problem embedded in the path equilibration algorithm for solving the traffic assignment problem.
We have implemented Dijkstra's algorithm, A* search and their bidirectional versions.
We also tested eight different priority queues for the shortest path algorithms.
Two strategies were developed to improve the performance of these shortest path algorithms when they are used in the iterative path equilibration algorithm.
The first strategy is to avoid the next few iterations
when the shortest path of the previous two iterations are the same.
The second strategy is to randomly skip the next shortest path calculation,
with the hope that the shortest path in the current and previous iteration is going to be the same.
In addition,
we have also investigated the possibility of using preprocessing methods such as A* search with landmarks.

\section{Conclusions}
We now conclude this project with the following results:
\begin{itemize}
    \item The priority queue implementation from the C++ standard template library has the best performance compared to the six implementations from the C++ Boost library and binary search tree from the C++ standard template library.
    \item A* search algorithm using zero-flow travel times as heuristic estimates has the best performance.
    \item Bidirectional versions of Dijkstra's algorithm and A* search have worse performances. Bidirectional Dijkstra is worse because it has to check the stopping criterion at each step. Bidirectional A* is worse because its search area is larger than that of unidirectional A*.
    \item The strategy of avoiding the next few iterations of shortest path calculations is viable when small number of iterations are skipped.
    \item The strategy of avoiding the shortest path calculations randomly is viable. By using A* search and 50\% random skipping on large networks that require millions of shortest path calculations in each iteration, the run times are further improved by about 25\% compared to just using A* search.
    %\item Preprocessing shortest path algorithms are not applicable for the traffic assignment problem, due to their long preprocessing time can slow down the whole traffic assignment process when the road network need to be changed frequently.
    %\item A* search with landmarks is not applicable unless we decide to concentrate on a specific road network, for example the Auckland Regional Transport model. 
\end{itemize}

\section{Future work}
\begin{itemize}
    \item In this project, only the path equilibration algorithm has been used to test the strategies of avoiding shortest path calculations.
        The strategies may also be applicable for other iterative algorithms that solve the traffic assignment problem.

    \item The current A* search algorithm only runs on a single thread.
        The algorithm can be improved by implementing a multi-threaded version developed
        by \citet{Inam}.
        The algorithm will run extremely fast as it is designed for GPGPU (General Purpose Graph Processing Unit) run on multi processors using many threads concurrently.
        The main modification of the algorithm is that instead of sequentially updating all out-going arcs from the labelled node,
        we update them in parallel using multiple threads.

    \item A* search with landmarks algorithm may be applicable for our problem but needs to be studied further.
        This algorithm is difficult to implement,
        this is because the number of landmarks to use and deciding their placements is another optimisation problem.
        Tests need to be done in order to find the best performance.

    \item A family of algorithms exists for dynamically changing graphs,
        including graphs that have moving nodes or changing arcs.
        One particular algorithm that tackles the problem of changing arc costs is the Lifelong Planning A* (LPA*), developed by \citet{LPA*}.
        \citet{LPA*} were able to show experimentally that LPA* is more efficient than A* if the change in arc costs are close to the destination.
        This means LPA* may be applicable for traffic assignment but more study needs to be done.
\end{itemize}

%\todoin[inline]{Run PE for a number of iterations, don't add any path until it is equilibrated}
%\todoin[inline]{one might also consider parallel shortest path runs in TA}
