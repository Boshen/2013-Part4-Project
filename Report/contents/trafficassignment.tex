\chapter{The Traffic Assignment Problem} \label{chap:ta}

In a transportation network,
every traveller wishes to travel between different pairs of origins and destinations.
As travellers start to travel in the network,
congestion happens and travel speeds tend to decrease rapidly due to more and more interactions between the travellers and an increase in traffic volume.
This leads to the problem of travellers wishing to find the fastest route to travel on,
meanwhile taking account of congestion while every other traveller is trying to do the same.
In transportation modelling point of view,
we wish to find a traffic flow pattern in the network with a given travel demand between the origin and destination pairs. 
This is called the Traffic Assignment Problem.

The Traffic Assignment Problem is commonly solved by traffic equilibrium models.
The notion of traffic equilibrium was first formalized by \cite{Wardrop},
where he introduced the postulate of the minimisation of the total travel costs.
His first principle states that ``the journey times on all routes actually used are equal and less than those which would be experienced by a simple vehicle of any unused route.''
%Under the assumption that the travellers have complete knowledge about the network and they always choose the best route to travel based on the current information about the network,
%The principle means that the traffic is in equilibrium when no traveller in the network can find a faster route than the one that is already being travelled on.
The traffic flows that satisfy this principle are referred to as ``user optimal'' flows,
as each traveller chooses the route which is the best for them.
On the other hand,
we can also find traffic flows that are ``system optimal'',
which is characterized by Wardrop's second principle, stating that ``the average journey time is minimum''.

In this chapter,
the network equilibrium model is first stated.
Then a particular solution algorithm for solving such a model is described.
Finally the motivation of this project in the context of solving the traffic assignment problem is brief explained.

\section{The network equilibrium model}
%In this section,
%the deterministic symmetric network equilibrium model for solving the user optimal of the traffic assignment is summarised.
In order to formulate a model for solving the traffic assignment problem,
two assumptions are made.
We assume deterministic traffic flows and fixed demand,
and the change of travel time on any link does not depend on the change of traffic flows on the other links.
These assumptions may not be realistic \todo{keep?} but they allow us to formulate the traffic assignment model as a mathematical program.
%The model assumes the link cost functions are separable, i.e.\ the function values are only dependent on the flows on the link itself, not other flows that do not belong to that link.
%The model also assumes the traffic demand of the network is fixed.

Using notations from \citet{Florian, Florian2008},
we consider a transportation network represented as a graph $G = (N, A)$,
where $N$ is a set of nodes and $A$ a set of directed links in the network.
The number of vehicles (or flow) on link $a$ is $v_a$ ($a \in A)$,
and the cost of travelling on a link is given by a link cost function $c_a(v)$ ($a \in A$),
where $v$ is the vector of link flows over the entire network.

Let $I$ be the set of origin-destination (O-D) pairs
and $K_i$ be the set of cycle-free paths for O-D pair $i \in I$.
The origin to destination traffic demands $g_i$ ($i \in I$) are distributed over directed paths $k \in K_i$,
and it is assumed $K_i \neq \emptyset$ and $K = \cup_{i \in I} K_i$.
Let $h_k$ be the flows on paths $k$ that satisfy the conservation of flow and non-negativity constraints:
\begin{align} \label{model_1}
    \sum_{k \in K_i} h_k & = g_i \quad \forall i \in I,  \\
    h_k &\geq 0 \quad \forall k \in K.
\end{align}
The corresponding link flows $v_a$ are given by:
\begin{equation}
    v_a = \sum_{k \in K} \delta_{ak} h_k, \quad \forall a \in A,
\end{equation}
where
\begin{equation} \label{model_4}
    \delta_{ak} = 
    \begin{cases}
        1 & \text{if link $a$ is on path $k$},\\
        0 & \text{otherwise}.
    \end{cases}
\end{equation}

With constraints \eqref{model_1} - \eqref{model_4},
the user optimal objective is
\begin{equation} \label{eq:user_optimal}
    \min S(v) = \sum_{a\in A} \int_0^{v_a} c_a(x) \mathrm{d} x.
\end{equation}

In order to solve for user equilibrium,
the \citep{Wardrop} user equilibrium condition is applied:

\begin{equation} \label{wardrop1}
    c_k(v^{\ast}) - u_i^{\ast} 
    \begin{cases}
        =0 & \text{if } h_k^{\ast} > 0 \\
        \geq 0 & \text{if } h_k^{\ast} = 0
    \end{cases}
    ,
    \quad \forall k \in K_i, i \in I,
\end{equation}
where 
\begin{equation}
    c_k(v) = \sum_{a \in A} \delta_{ak} c_a(v), \quad \forall k \in K
\end{equation}
and
\begin{equation} \label{wardrop3}
    u_i = \min_{k \in K_i} \left[ c_k(v) \right], \quad \forall i \in I.
\end{equation}

To elaborate, this condition means that the traffic is in equilibrium when no traveller in the network can find a faster route than the one that is already being travelled on.
Furthermore, this condition assumes that the travellers have complete knowledge about the network,
they always choose the best route to travel based on the current information about the network.
This means the equilibrium is the result of everybody simultaneously attempting to minimize their own travel times.

The \citet{BPR} (BPR) link cost function
$c_a(v_a)$ is often used to model the travel time on link $a \in A$.
The function is given by
\begin{equation}
    c_a(v_a) = c_f \left(1 + B \left( \frac{v_a}{C_a} \right)^\alpha \right)
\end{equation}
where $B$ and $\alpha$ are the parameters for the level of congestion that can be experienced,
and $c_f$ and $C_a$ are the free-flow travel time and link capacity.
It is important to note this cost function only depends on traffic flow on its own link, and it is strictly monotonic, continuous and differentiable.
An example of this link cost function is shown in Figure~\ref{fig:flowfunction}

\begin{figure}[H]
    \centering
    \begin{tikzpicture}
        \begin{axis}
            [
                domain=0:10000,
                black, no markers, smooth,
                xmin=0,xmax=10000,
                ymin=0,ymax=130,
                axis x line=bottom,
                axis y line=left,
                xlabel={Traffic Flow},
                ylabel={Travel Time},
                every axis y label/.style={at={(current axis.above origin)},anchor=north east, xshift=-2pt},
                every axis x label/.style={at={(current axis.right of origin)},anchor=north west, xshift=-2pt},
                extra y ticks={20},
                extra y tick labels={Freeflow time $(c_f)$},
                extra y tick style={overlay},
                xtick=\empty,
                ytick=\empty,
                xticklabel=\empty,
                yticklabel=\empty,
                scaled ticks=false,
                extra x ticks={9000},
                extra x tick style={xticklabel pos=right, xticklabels={Capacity $(C_a)$}, xmajorgrids=true}
            ]
            \addplot [->, black] {20+0.15*(x/2000)^4}; 
        \end{axis}
    \end{tikzpicture}
    \caption{Travel time function.}
    \label{fig:flowfunction}
\end{figure}

\section{The path equilibration algorithm}
The deterministic symmetric network equilibrium model described in the previous section is equivalent to a convex cost differentiable optimization problem,
where a wide range of algorithms exists for solving such problems.
They include, the linear approximation method,
the linear approximation with parallel tangents method,
the restricted simplicial decomposition method,
and the path equilibration algorithm,
see \citet{Florian}.

For this project,
we focus on the path equilibration algorithm.
\todo[inline]{why? reformat to a new section for solving TAP}
The general approach of the algorithm is equivalent to
Gauss-Seidel decomposition (an iterative method for solving a linear system of equations).
In a step of the algorithm,
path flows $h_k \forall k \in K_i$ between a single O-D pair $i$ are considered by keeping the flows of all other O-D pairs fixed.
The algorithm iteratively solves each O-D pair sub-problem until any of them cannot be improved.

The sub-problem for solving each O-D pair $i$ is another fixed-demand network equilibrium problem.
\begin{align} \label{eq:sb1}
    \min & \quad \sum_{a \in A} \int_0^{v_a^i + \bar{v}_a} c_a(x) \mathrm{d} x \\
    \text{s.t.} &\quad \sum_{k \in K_i} h_k = \bar{g}_i, \quad i \in I, \\
    & \quad h_k \geq 0, \quad k \in K_i,
\end{align}
where flows $\bar{v}_a$ are fixed:
\begin{equation}
    \bar{v}_a = \sum_{i' \neq i} \sum_{k \in K_i} \delta_{ak} h_k,
\end{equation}
and flows $v_a^i$ are varied:
\begin{equation} \label{eq:sb5}
    v_a^i = \sum_{k \in K_i} \delta_{ak} h_k.
\end{equation}

The Gauss-Seidel decomposition (or `cyclic decomposition' by O-D pair) is stated as follows.

\begin{table}[H]
    \begin{tabular}{ m{0.8\textwidth} }
        \hspace{-.5cm}\emph{`cyclic decomposition' by O-D pair \citep{Florian}} \\
        \emph{Step 0.} Given initial solution, set $l = 0$, $l' = 0$.\\
        \emph{Step 1.} If $l' = |I|$, stop; otherwise set $l = l \text{ mod } |I| + 1$ and continue.\\
        \emph{Step 2.} If the current solution is optimal for the $i$th sub-problem (\ref{eq:sb1})-(\ref{eq:sb5}), set $l' = l' + 1$ and return to step 1; otherwise solve the $l$th sub-problem, update the path flows, set $l' = 0$ and return to step 1.\\
    \end{tabular}
\end{table}

The path equilibration algorithm for solving \eqref{eq:sb1}-\eqref{eq:sb5} 
finds the solution of the traffic assignment problem by balancing path flows between each O-D pair.
One such algorithm, proposed by \citet{Dafermos}, 
finds the shortest and longest path and equalizes the travel times between them.
Let $K_i^{+} = \left\{ k \in K_i | h_k > 0 \right\}$ be the set of paths with positive flows.

The algorithm for solving each O-D pair $i$ is stated as follows.
\begin{table}[H]
    \begin{tabular}{ m{0.8\textwidth} }
        \hspace{-.5cm}\emph{Path Equilibration Algorithm \citep{Florian}} \\
        \emph{Step 0}. All-or-nothing assignment: find an initial solution $v_a^i$; $c_a = c_a(v_a^i+\bar{v}_a)$ and the initial $K_i^+$.\\
        \emph{Step 1}. Compute the costs of the currently used paths $c_k$, $k \in K_i^+$. Find $k_1$ such that $c_{k_1} = \displaystyle \min_{k \in K_i} \left[ c_k \right]$ and $k_2$ such that $c_{k_2} = \displaystyle \max_{k \in K_i^+} \left[c_k \right]$.\\
        If $(c_{k_2} - c_{k_1}) \leq \epsilon$, go to step 4;
        otherwise define the direction $d_{k_1} = (c_{k_2} - c_{k_1})$ for path flow $k_1$ and $d_{k_2} = (c_{k_1} - c_{k_2})$ for path flow $k_2$\\
        \emph{Step 2}. Find the step size $\lambda$ which redistributes the flow $h_{h_1} + h_{k_2}$ between the paths $k_1$ and $k_2$ in such a way that their costs become equal, that is, solve
    \end{tabular}
\end{table}
\vspace{-3em}
\begin{align}
    \min_\lambda & \quad \sum_{a \in A} \int_0^{v_a^i + \lambda y_a + \bar{v}_a} c_a(x) \mathrm{d}x \\
    \text{s.t.} & \quad 0 \leq \lambda \leq \left( \frac{-h_{k_2}}{d_{k_2}} \right), \\
    \text{where} & \quad y_a = \delta_{ak_1} d_{k_1} - \delta_{ak_2} d_{k_2}.
\end{align}
\vspace{-2em}
\begin{table}[H]
    \begin{tabular}{ m{0.8\textwidth} }
        \emph{Step 3}. Using the $\lambda$ obtained, update $h_k = h_k + \lambda d_k, k = \left\{ k_1, k_2 \right\}; v_a^i = v_a^i + \lambda y_a; c_a = c_a(v_a^i + \bar{v}_a)$. \\
        \emph{Step 4}. Compute the shortest path $\tilde{k}$ with cost $\tilde{s}_k = \displaystyle \min_{k \in K_i} \left[ c_k \right]$;
        if $\tilde{s}_k < \displaystyle \min_{k \in K_i^+} \left[ c_k \right]$, then the path $\tilde{k}$ is added to the set of kept paths, $K_i^+ = K_i^+ \cup \tilde{k}$ and return to step 1; otherwise stop.
    \end{tabular}
\end{table}
\todo[inline]{fix this, see page 8 Andrea fix}

In step 0, a so called all-or-nothing assignment is performed for each of the O-D pairs,
where it finds the shortest path and assigns all traffic flows along that path.
In step 1 and 2, the algorithm finds the two paths that have the minimum and maximum cost,
and balances the flow between them to equalize their costs.
These two steps are equivalent to solving the Wardrop equilibrium shown in Equations~\eqref{wardrop1}-\eqref{wardrop3}.
In step 4, the shortest path between the O-D pair is computed and added to the set of used paths for the all-or-nothing assignment and Wardrop equilibrium.
\todo[inline]{rewrite}

Now if we are given a large network and assume it requires many iterations to find the optimal solution,
it can be shown that a large number of shortest path calculations are needed.
For example, 
given the algorithm takes 20 iterations to solve a small network with $100,000$ O-D pairs,
it can take more than 6 minutes to solve if
each shortest path calculation consumes 0.01 second.
In reality,
networks may contain millions of O-D pairs.
Thus for this project we wish to
investigate and find a faster shortest path algorithm,
and also investigate other ways of improving it for the traffic assignment problem.

\section{Convergence and stopping criterion} \label{sec:convergence}
For completeness, the convergence criterion for the traffic assignment is discussed in this section.
It it known that the objective function of the problem (Equation~\ref{eq:user_optimal} is convex (any local minimum is also the global minimum),
so the convergence of the Gauss-Seidel strategy is ensured \citep{Florian2008}.
The convergence criterion of traffic assignment algorithms is normally measured by so called relative gap \citep{Rose}.
\cite{Rose} states that ``the relative gap is expressed by the difference between the current value of 
the objective function and the lower bound as a percentage of the current objective function.''
Here the objective function is the user equilibrium (UE) solution of the traffic assignment problem.
And the lower bound refers to the all-or-nothing (AON) assignment in step 0 of the path equilibrium algorithm.

The relative gap (RGAP) is computed as
\begin{align}
    \text{RGAP} &= \frac{\text{UE} - \text{AON}}{\text{UE}} \\
                &= \frac{\sum_{a \in A} v_a c_a - \sum_{i \in I} g_i c_{\tilde{k}_i}}{\sum_{a \in A} v_a c_a}.
\end{align}
The AON solution is the sum of travel times $c_{\tilde{k}_i}$ of each traffic demand $g_i$ travelling on their shortest path $\tilde{k}_i$.
The UE solution is the sum of travel times over the entire network.
These two values are improved from iteration to iteration during the path equilibration algorithm until they become identical, i.e.\ RGAP is converging towards 0.
The traffic assignment problem is solved when RGAP is 0,
but normally we stop the algorithm at some tolerance such as $10^{-6}$.
\todo[inline]{it doesn't decrease monotonically though\ldots}

It is worth to notice that the speed of convergence is highly dependent on how small we set the relative gap,
as well as the size and complexity of the network and number of supply and demand nodes.
A smaller relative gap will result in more iterations for the traffic assignment algorithms.
